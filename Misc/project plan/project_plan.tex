% project plan for energy subsystem
%-----------------------------------------------------------------------
\documentclass[a4paper, 12pt]{report}

% Packages
%-----------------------------------------------------------------------
\usepackage{graphicx}
\usepackage{float}
\usepackage{hyperref}
\usepackage{tabularx}
\usepackage{booktabs}
\usepackage{pdfpages}
\usepackage{pdflscape}
\usepackage{rotating}
\usepackage{listings}
\usepackage{color}
\usepackage{amsmath}
\usepackage{subcaption}
\usepackage{wrapfig}
\usepackage{caption}
\usepackage{longtable}
\usepackage{multirow}

% Page settings
%-----------------------------------------------------------------------
\setlength{\parindent}{0pt}
\setlength{\parskip}{1em}
\graphicspath{{./img/}}
\setcounter{tocdepth}{2}

% Document information
%-----------------------------------------------------------------------
\title{Project Plan Energy subsystem \\ \small{Specialization project and masters project 2023-2024}}
\author{Daniel Carton, Tobias Lien}
\date{\today}

% Document
%-----------------------------------------------------------------------
\begin{document}

\maketitle

\tableofcontents
\newpage

\section{Introduction}
This paper will act as a preliminary project plan prior to any work being done on any of the projects. The plan will be updated as the projects progress and more information is available. For now, this plan represents more the expectations and goals of the projects rather than a detailed plan of how to achieve them. \\

The projects discussed in this paper are two individual projects that dont necessarily have a strict relation to each other, but benefit from being grouped together. The specialization project is a project done in the start of the last year of the MSELSYS master's programme, while the master project is done in the end of the same year. The reason these projects benefit from being grouped together, is that the specialization project can be looked at as a precourser to the master project. The specialization project will be used to explore and prototype solutions to the problems that can be solved more in-depth in the master project. \\




\chapter{Specialization project}

\section{Objective and motivation} % Why are we doing this?
\section{Deliverables}          % What are we doing?
\section{Scope}                % What are we not doing?
\section{Tasks and responsibilities} % Who is doing what?
\section{Timeline and milest ones}   % When are we doing what?


\chapter{Master project - Early scope}

\section{Objective and motivation} % Why are we doing this?
\section{Scope}                % What are we not doing?

\chapter{Project Comparisons}
\begin{table}[h!]
    \begin{tabular}{|l|l|l|l|}
        \hline
        \textbf{Aspect [Early Targets]} & \textbf{specialization project} & \textbf{master project} & \textbf{notes} \\
        \hline
        \multicolumn{4}{|c|}{\textbf{Energy storage}} \\
        \hline
        Storage efficiency & $\geq 75\%$ &  $\geq 90\%$ & \\
        \hline
        Storage capacity & $\geq 69 \mu Ah$ & $\geq  420 \mu Ah$ & \\
        \hline
        Output voltage & $ 1.8 V$ & $ 1.8 V$ & Will depend on\\
        &&& connected SoC\\
        &&& circuit \\
        \hline
        \multicolumn{4}{|c|}{\textbf{Data Collection}} \\
        \hline
        Sample rate & $8\times 1 kHz$ & $8\times 1 MHz$ & \\
        \hline
        Sample resolution & $16 bit$ & $24 bit$ & \\
        \hline
        \multicolumn{4}{|c|}{\textbf{Data Processing}} \\
        \hline
        Data transmission & Rs-232 & USB high-speed & Requirement \\
        protocol &&& as result of\\
        &&& data collection \\
        &&& rate and volume\\
        \hline
        Data rate & $  \approx 192 kbps$ & $\approx 200 Mbps$ & \\
        \hline
        Data storage & Connected PC & Connected PC & \\
        \hline
    \end{tabular}
\end{table}

\end{document}